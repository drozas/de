\section*{Introducci�n}
\noindent
{\Large \bfseries E}n un lugar de la Mancha, de cuyo nombre no quiero acordarme, no ha mucho tiempo que viv�a un hidalgo de los de lanza en astillero, adarga antigua, roc�n flaco y galgo corredor. 
Una olla de algo m�s vaca que carnero, salpic�n las m�s noches, duelos y quebrantos los s�bados, lantejas los viernes, 
{\it alg�n palomino de a�adidura los domingos}, consum�an las tres partes de su hacienda. 
El resto della conclu�an sayo de velarte, calzas de velludo para las fiestas, con sus pantuflos de lo mesmo, y los d�as de entresemana se honraba con su vellor� de lo m�s fino. 
Ten�a en su casa una ama que pasaba de los cuarenta, y una sobrina que no llegaba a los veinte, y un mozo de campo y plaza, 
{\footnotesize que as� ensillaba el roc�n como tomaba la podadera.}