\section{Listas formateadas}
\noindent
En esta segunda parte se va a trabajar con diferentes estilos de listas
formateadas.\par
Se comienza con una lista de entorno {\it itemize} donde se van cambiando los
s�mbolos:

\begin{itemize}
  \item Primer �tem con s�mbolo por defecto.
  \renewcommand{\labelitemi}{$\top$}
  \item Segundo item con s�mbolo diferente.
  \renewcommand{\labelitemi}{$\rightarrow$}
  \item Tercer item con s�mbolo diferente.
  \renewcommand{\labelitemi}{$\clubsuit$}
  \item Cuarto item con s�mbolo diferente.
\end{itemize}
A continuaci�n, listas anidadas con diferentes entornos.

\begin{itemize}
  %S�lo tiene en cuenta los niveles de itemize
  \renewcommand{\labelitemi}{$\bullet$}
  \renewcommand{\labelitemii}{$-$}
  \item Primer item de la lista m�s externa.
	\begin{enumerate}
	  	\item Primer item de la enumeraci�n anidada.
	  	\begin{description}
 			\item[Caso 1: ] Primer caso de la anidaci�n sobre la enumeraci�n.
 			\begin{enumerate}
   				\item Primer item enumerado de la descripci�n.
   				\item Segundo item enumerado de la descripci�n.
			\end{enumerate}
 			\item[Caso 2:] Segundo caso de la anidaci�n sobre la enumeraci�n.
 			 	\begin{itemize}
   					\item Primer item anidado sobre el segundo caso.
				\end{itemize}
		\end{description}
	  	\item Segundo item de la enumeraci�n anidada.
    \end{enumerate}
  \item Segundo item de la lista m�s externa.
\end{itemize}
