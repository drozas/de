%%This is a very basic article template.
%%There is just one section and two subsections.
\documentclass[titlepage]{book}
\usepackage[latin1]{inputenc}
\usepackage[spanish]{babel}

\usepackage{fancyhdr}
\pagestyle{fancy}
\fancyhf{} % borrar todos los ajustes
% En lo siguiente, fancyhead sirve para configurar la cabecera, fancyfoot para
% configurar el pie de p�gina.
% Justificaci�n: C=centered, R=right, L=left, (nada)=LRC
% P�gina: O=odd, E=even, (nada)=OE
\fancyhead[RO]{Documentaci�n electr�nica} 
\fancyhead[LE]{Pr�ctica 7}
\fancyfoot[C]{\thepage}



\title{ {\Large {\bf Pr�ctica 7 \LaTeX \\}Documentaci�n Electr�nica\\}
{\small Curso 2008/2009}}
%Declaramos fecha pero la dejamos vac�a, porque si no mete una autom�ticamente
\date{}


\begin{document}

%%Por definici�n los llama cuadros, renombramos
\renewcommand{\tablename}{Tabla}
\renewcommand{\listtablename}{Lista de Tablas}

\maketitle
\tableofcontents
\listoftables


\chapter{Introducci�n}
En esta pr�ctica, los contenidos que se van a ver son los siguientes:
\begin{itemize}
	\item Tablas
	\item Indices de contenidos, de tablas y de figuras
	\item Bibliograf�a en \LaTeX
\end{itemize}
Al igual que se hizo en la pr�ctica 6, se trata de crear este mismo documento
que describe el contenido de la pr�ctica.

\chapter{Tablas}
\noindent
En algunos documentos puede ser necesario incluir una tabla que ilustre o aclare
nuestra narraci�n.\par
El entorno que permite describir tablas es:\newline

\verb|{\begin{tabular}[caracter�sticas de la tabla]...\end{tabular}|
%\begin{verbatim}
%{\begin{tabular}[caracter�sticas de la tabla]...\end{tabular}
%\end{verbatim}
, donde\newline

\verb|[caracter�sticas de la tabla]|podr� contener lo siguiente:

\begin{itemize}
	\renewcommand{\labelitemi}{$\bullet$}
	\item ``\verb|l|'', para describir una columna con texto justicado a la
	izquierda
	\item ``\verb|r|'', para describir una columna con texto justicado a la derecha
	\item ``\verb|c|'', para describir una columna con texto centrado
	\item ``\verb|p{ancho}|'', para una columna que contenga texto con saltos de
	l�nea
	\item ``\verb|\|'', para una columna que queremos que vaya entre una l�nea
	vertical
\end{itemize}

Cada filla de una tabla se separa de la siguiente mediante una indicaci�n de
salto de l�nea ``\verb|\\|".\par
Los datos de una columna se separan de los de la siguiente utilizando el
car�cter ``\&''.\par
Veamos un ejemplo:\newline

\begin{tabular}{lcc}
Descriptive levels & Precision & Recall \\
General & 99.7\% & 99.4\% \\
Document specific & 96.8\% & 96.3\% \\
Proper noun & 94.4\% & 99.1\% \\
\end{tabular}
\newline

El mismo ejemplo con separadores de l�nea vertical entre las columnas:\newline

\begin{tabular}{|l|c|c|}
Descriptive levels & Precision & Recall \\
General & 99.7\% & 99.4\% \\
Document specific & 96.8\% & 96.3\% \\
Proper noun & 94.4\% & 99.1\% \\
\end{tabular}
\newline

Si queremos separar unas filas de otras mediante l�neas horizontales, debemos
utilizar \verb|\hline|. \newline

\begin{tabular}{|l|c|c|}
\hline
Descriptive levels & Precision & Recall \\
General & 99.7\% & 99.4\% \\
Document specific & 96.8\% & 96.3\% \\
Proper noun & 94.4\% & 99.1\% \\
\hline
\end{tabular}
\newline

M�s l�neas
\newline

\begin{tabular}{|l|c|c|}
\hline
Descriptive levels & Precision & Recall \\
\hline
General & 99.7\% & 99.4\% \\
\hline
Document specific & 96.8\% & 96.3\% \\
\hline
Proper noun & 94.4\% & 99.1\% \\
\hline
\end{tabular}
\newline

Algunas l�neas m�s
\newline

\begin{tabular}{||l|c|c||}
\hline \hline
Descriptive levels & Precision & Recall \\
\hline \hline
General & 99.7\% & 99.4\% \\
\hline
Document specific & 96.8\% & 96.3\% \\
\hline
Proper noun & 94.4\% & 99.1\% \\
\hline \hline
\end{tabular}
\newline

Queremos una tabla centrada
\newline
\begin{center}
\begin{tabular}{||l|c|c||}
\hline \hline
Descriptive levels & Precision & Recall \\
\hline \hline
General & 99.7\% & 99.4\% \\
\hline
Document specific & 96.8\% & 96.3\% \\
\hline
Proper noun & 94.4\% & 99.1\% \\
\hline \hline
\end{tabular}
\end{center}



Para agrupar varias columnas en una se utiliza \verb|\multicolumn|. En el
siguiente ejemplo se agrupan dos columnas en una:
\newline

\begin{tabular}{||l||c|c||c|c||} \hline\hline
& \multicolumn{2}{c||}{Spanish} &
\multicolumn{2}{c||}{English}\\ \hline
Descriptive levels & Precision & Recall & Precision & Recall \\
\hline\hline General & 99.7\% & 99.4\% & 98.6\% & 98.5\% \\
\hline Document specific & 96.8\% & 96.3\% & 95.8\% & 95.6\% \\
\hline Proper noun & 94.4\% & 99.1\% & 98.8\% & 99.8\% \\ \hline
\hline
\end{tabular}
\newline

Las tablas que se han definido hasta ahora aparec�aan en la posici�n en la
que se han escrito en el documento \verb|.tex|.

\section{Referencias a tablas}
\noindent
Sin embargo, en ocasiones puede no interesar incluir una tabla en el discurso de
una narraci�n, sino hacer una referencia a la tabla que puede estar ubicada en
otra parte del documento. 
En este caso la tabla (entorno tabular)deber�a estar incluida dentro de un
entorno \verb|\begin{table}[ubicacion]...\end{table}| como en el ejemplo
siguiente:

\begin{verbatim}
\begin{table}[h]
\small \centering
\begin{tabular}{||l||c|c||c|c||} \hline\hline
& \multicolumn{2}{c||}{Spanish} &
\multicolumn{2}{c||}{English}\\ \hline
Descriptive levels & Precision & Recall & Precision & Recall \\
\hline\hline General & 99.7\% & 99.4\% & 98.6\% & 98.5\% \\
\hline Document specific & 96.8\% & 96.3\% & 95.8\% & 95.6\% \\
\hline Proper noun & 94.4\% & 99.1\% & 98.8\% & 99.8\% \\ \hline
\hline
\end{tabular}
\end{table}
\end{verbatim}

\begin{table}[h]
\small \centering
\begin{tabular}{||l||c|c||c|c||} \hline\hline
& \multicolumn{2}{c||}{Spanish} &
\multicolumn{2}{c||}{English}\\ \hline
Descriptive levels & Precision & Recall & Precision & Recall \\
\hline\hline General & 99.7\% & 99.4\% & 98.6\% & 98.5\% \\
\hline Document specific & 96.8\% & 96.3\% & 95.8\% & 95.6\% \\
\hline Proper noun & 94.4\% & 99.1\% & 98.8\% & 99.8\% \\ \hline
\hline
\end{tabular}
\end{table}

\par
Se ha intentado que LATEX la muestre en la posici�n en la que la se ha
escrito mediante \verb|\begin{table}[h]|. Si no hubiese cabido en esta parte de
la p�gina, \LaTeX la habr�a ubicado en la siguiente p�gina. Los valores del
argumento de ubicaci�n pueden ser:

\begin{itemize}
	\renewcommand{\labelitemi}{$\bullet$}
	\item \verb|"h":| aqu� o lo m�s pr�ximo posible donde se encuentra declarada
	\item \verb|"t":|'' en la parte superior de una p�gina
	\item \verb|"p":|'' en la parte inferior de una p�gina
	\item \verb|"b":|'' en una p�gina que s�lo contenga tablas y figuras

\end{itemize}

En estos casos conviene que en la narraci�n se haga referencia a la tabla.
Esto se hace con \verb|~\ref{tabla1}|, escribiendo dentro del entorno de la
tabla \verb|\label{tabla1}|.
%%Esto es una referencia a la tabla: ~\ref{tabla1}

\begin{verbatim}
\begin{table}
\begin{tabular}{||l||c|c||c|c||} \hline\hline
& \multicolumn{2}{c||}{Spanish} &
\multicolumn{2}{c||}{English}\\ \hline
Descriptive levels & Precision & Recall & Precision & Recall \\
\hline\hline General & 99.7\% & 99.4\% & 98.6\% & 98.5\% \\
\hline Document specific & 96.8\% & 96.3\% & 95.8\% & 95.6\% \\
\hline Proper noun & 94.4\% & 99.1\% & 98.8\% & 99.8\% \\ \hline
\hline
\end{tabular}
\label{tabla1}
\caption{Results of description levels encoding}
\end{table}
\end{verbatim}


\par
Suele ser conveniente que la tabla tenga un t�tulo (pie de tabla). Esto
permite que se aclare su contenido independientemente de donde aparezca. El
t�tulo se a�nade con \verb|\caption{...}|.
\par

Se puede ver un ejemplo en la tabla ~\ref{tabla1}
%%Duda: Problema con posicionamiento de esta tabla. He probado con otro
% valores, pero no hay manera.
\begin{table}[h]
\begin{tabular}{||l||c|c||c|c||} \hline\hline
& \multicolumn{2}{c||}{Spanish} &
\multicolumn{2}{c||}{English}\\ \hline
Descriptive levels & Precision & Recall & Precision & Recall \\
\hline\hline General & 99.7\% & 99.4\% & 98.6\% & 98.5\% \\
\hline Document specific & 96.8\% & 96.3\% & 95.8\% & 95.6\% \\
\hline Proper noun & 94.4\% & 99.1\% & 98.8\% & 99.8\% \\ \hline
\hline
\end{tabular}
\label{tabla1}
%\begin{center}
\caption{Results of description levels encoding}
%\end{center}
\end{table}

\par
Las referencias no s�lo se utilizan con tablas, sino que se puede hacer
referencia a una figura, secci�n o cap�tulo de un documento.


\chapter{�ndices}

\section{�ndice de contenidos}
\noindent
La creaci�n de �ndices en \LaTeX es totalmente autom�tica, lo cual supone una
gran ventaja ya que el usuario no tiene la necesidad de ir revisando los n�meros
de p�gina de cada capitulo.
Para generar un �ndice de contenidos se utiliza el comando
\verb|\tableofcontents| en el punto del documento donde se quiere que aparezca
el �ndice. Ser�a necesario compilar el fichero \verb|.tex| dos veces para que se
enlace bien y genere de forma correcta el �ndice.

\section{�ndice de figuras y tablas}
Para generar un �ndice de figuras se utiliza el comando \verb|\listoffigures| y
para generar un �ndice de tablas el comando \verb|\listoftables|. Se da la
orden en el punto donde se quiere que aparezca el �ndice correspondiente.
Un ejemplo t�pico para generar �ndices en las p�ginas iniciales es el
siguiente:

\begin{verbatim}
\begin{document}
\maketitle
\newpage % Salto de p�gina
\tableofcontents % indice de contenidos
\newpage % Salto de pagina
\listoffigures % Indice de figuras
\newpage % Salto de pagina
\listoftables % Indice de tablas
\newpage % Salto de pagina
...
\end{document}
\end{verbatim}
Como en el cap�tulo 2 se ha creado al menos una tabla con t�tulo se pide
generar un �ndice de tablas que tendr�a una �nica entrada.
%%No se agrega autom�ticamente, pero consta en el enunciado
%\newpage

\chapter{Variaciones en el encabezado y en el pie de p�gina}
\noindent
Por defecto hay un estilo para el encabezado y pie de p�gina. Por ejemplo,
en esta pr�ctica se puede ver que por defecto aparece un encabezado para las
p�ginas en may�sculas. Pero, �si lo queremos en min�sculas por ejemplo, qu�
se deber�a hacer?
Existe un paquete llamado \verb|fancyhdr| que permite que podamos configurar
tanto el encabezado como el pie de p�gina a nuestro gusto. Se utilizar�a el
paquete como sigue:

\begin{verbatim}

\usepackage{fancyhdr}
\pagestyle{fancy} ...
\fancyhf{} % borrar todos los ajustes
% En lo siguiente, fancyhead sirve para configurar la cabecera, fancyfoot para
% configurar el pie de p�gina.
% Justificaci�n: C=centered, R=right, L=left, (nada)=LRC
% P�gina: O=odd, E=even, (nada)=OE
\fancyhead[RO,LE]{Cabecera1} \fancyhead[LO,RE]{Cabecera2}
\fancyfoot[LO,CE]{Pie1} \fancyfoot[RO,CE]{Pie2} ...

\end{verbatim}

Si en alguna p�gina espec�fica se prefiere aplicar un estilo concreto se puede
usar \verb|\thispagestyle{arg}|, con \verb|arg={fancy}| o \verb|arg={plain}| o
\verb|arg={empty}|, dependiendo de si se quiere aplicar el estilo especial, el
estilo por defecto o ninguno (sin cabecera ni pie), respectivamente.
Con ``\verb|\Cabecera1|'', ``\verb|\Cabecera2|'', ``\verb|\Pie1|'', ``\verb|\Pie2|'', ... se
puede especificar el n�mero/nombre del cap�tulo/secci�n, etc. Para ello, hay
que tener en cuenta que:

\begin{itemize}
	\renewcommand{\labelitemi}{$\bullet$}
	\item \verb|\leftmark| informaci�n de nivel superior (por ejemplo, cap�tulo en la clase
de documento \verb|book|)
	\item \verb|\righttmark| informaci�n de nivel inferior (p.e.: secci�n en clase
	book)
\end{itemize}

Estos comandos se introducen en \verb|\fancyhead| o \verb|\fancyfoot| seg�n se
requiera, por ejemplo, en un documento clase book,
\verb|\fancyhead[LO,RE]{\leftmark}| indica que debe aparecer el nombre del
cap�tulo en la parte izquierda de la cabecera si es p�gina impar, y en la
derecha si es p�gina par. Para controlar c�mo se representan los cap�tulos,
secciones, etc., en la cabecera/pie del documento, se redefinen los comandos
\verb|\chaptermark|,\verb|\sectionmark|,\verb|\subsectionmark|, etc. tras la
llamada a \verb|\pagestyle{fancy}|, por ejemplo: El n�mero de p�gina es
\verb|\thepage|. Puede aparecer en \verb|\fancyhead| o \verb|\fancyfoot|, seg�n
se quiera; por ejemplo, \verb|\fancyfoot[C]{\thepage}| indica que el n�mero de
p�gina aparecer�a centrado en el pie de todas las p�ginas. En nuestro caso
queremos que en el encabezado, en las p�ginas impares, en la parte izquierda
aparezca ``Pr�ctica 7'' y en las p�ginas pares, en la parte derecha
aparezca ``Documentaci�n Electr�nica''. Adem�s, los n�meros de p�gina
aparecer�an en el pie de p�gina y centrados. El resultado de estos cambios se
puede ver en el archivo ``practica7Variacion.pdf''. Escribid la sintaxis para
conseguir lo anterior y comprobar que el encabezado y pie de p�gina ha cambiado.


\chapter{Bibliograf�a}
\noindent

Se puede generar bibliograf�a de dos formas diferentes: haciendo uso del entorno
\verb|\begin{thebibliography}| o bien usando el programa BibTeX.
La primera opci�n es m�s b�sica y, por tanto, menos flexible y potente,
aunque quiz� pueda resultar m�s sencilla. La sintaxis es la siguiente:

\begin{verbatim}
\begin{thebibliography}{longitud}
...
\bibitem [Leyenda]{Etiqueta} Texto
...
\end{thebibliography}
\end{verbatim}

donde \verb|longitud| es una cadena de anchura mayor o igual a la m�xima que
va a ser utilizada en la numeraci�n; leyenda es un par�metro opcional que
representa la identificaci�n que se colocar�a en el texto en el lugar donde se
ubique la cita (\LaTeX tiene una identificaci�n num�rica por defecto);
\verb|etiqueta| es la cadena que se usa para referirse a un item concreto de la
lista bibliogr�fica y, por �ltimo, \verb|texto| es la cita en s� (autor, t�tulo,
editorial, etc.), con el formato y orden que se quiera darle.
\par
Para citar un elemento de la bibliograf�a desde cualquier parte del documento
se usa el comando \verb|\cite{etiqueta}|. A continuaci�n se muestra un ejemplo:

\begin{verbatim}
\begin{thebibliography}{99}
\bibitem{Hahn} Hahn, J.``\LaTeX $\,$ for eveyone''.
Prentice Hall, New Jersey, 1993.
\end{thebibliography}
\end{verbatim}

La referencia al libro anterior ser�a:

\begin{verbatim}
	En \cite{Hahn} se pueden ver los aspectos relativos al ``Picture Environment''
\end{verbatim}

Donde el resultado es:

\begin{verbatim}
	En [1] se pueden ver los aspectos relativos al "Picture Environment".
\end{verbatim}

Con el entorno anterior el autor tiene completa libertad para definir el formato
de la bibliograf�a. Esto puede ser ventajoso pero, por otra parte, puede
resultar un inconveniente si se quiere por ejemplo reordenar los items. Como
alternativa a este entorno se puede utilizar el programa BibTeX, que se integra
con \LaTeX para generar, de forma autom�tica, un entorno \verb|thebibliography|
siguiendo una serie de reglas de formato establecidas.
Para usar BibTeX hay que definir la base de datos de citas bibliogr�ficas que
se quiere incluir en el documento. Para ello se edita un fichero con extensi�n
\verb|.bib| que tiene un formato concreto (contiene una serie de registros
especiales, uno por referencia). Hay varios tipos de registros posibles, uno por cada tipo
de referencia que se puede incluir, por ejemplo los siguientes:

\begin{verbatim}
@BOOK{LibroLatex,
author = {{Cascales Salinas, Bernardo} and others},
publisher = {ADI},
title = {\LaTeX{}, una imprenta en sus manos},
year = {2000}
}

@ARTICLE{armistice,
author = {{Cabrero, David}, {Abalde, Carlos},
{Varela, Carlos} and {Castro, Laura}},
title = {ARMISTICE: An Experience Developing
Management Software with Erlang},
journal = {Principles, Logics, and Implementations of
High-Level Programming Languages (PLI'03)},
month = {Agosto},
year = {2003}
}
\end{verbatim}

Las citas se incluyen en el texto utilizando el comando \verb|\cite{etiqueta}|,
con la particularidad de que s�lo aquellos documentos de la base de datos que
sean mencionados en un \verb|\cite| se visualizar�n en la versi�n compilada
del documento. Para visualizar alg�n documento no referenciado hay que indicarlo con
el comando \verb|\nocite{etiqueta}|. Si se desea que aparezcan todos los
elementos presentes en la base de datos se puede emplear \verb|\nocite*|.
Una vez creado el fichero de bibliograf�a su uso se indica incluyendo las
siguientes l�neas antes de la finalizaci�n del documento
(\verb|\end{document}|):

\begin{verbatim}
\bibliography{ficherobase}
\bibliographystyle{estilo}
\end{verbatim}

donde ficherobase es el nombre del fichero de la base de datos sin la extensi�n
\verb|.bib| y estilo es uno de los siguientes:

\begin{itemize}
	\renewcommand{\labelitemi}{$\bullet$}
	\item \verb|plain| Ordena las entradas alfab�ticamente y las numera. El orden
	que establece es: autor, a�o, t�tulo.
	\item \verb|usrt| Igual que plain pero las entradas se ordenan por orden de
	citaci�n en el documento.
	\item \verb|alpha| A cada entrada le asigna una etiqueta basada en el nombre
	del autor y el a�o de publicaci�n. El orden que establece es: etiqueta, autor, a�o y t�tulo.
	\item \verb|abbrv| Equivalente a \verb|plain|, hace que las entradas sean m�s
	peque�as al abreviar los nombres de los autores, meses y nombres de las revistas.
\end{itemize}


Una vez hecho esto, es necesario compilar la bibliograf�a utilizando la herramienta
\verb|bibtex|. Este programa recibe como argumento un fichero \verb|.aux|,
procedente de una primera compilaci�n del documento LATEX, en la que se
identifican todas las citas bibliogr�ficas a las que se ha hecho referencia.
Tras el procesado con \verb|bibtex| ser� necesaria una nueva compilaci�n con
latex para que queden resueltas:

\begin{verbatim}
$ latex midocumento(.tex)
$ bibtex midocumento(.aux)
$ latex midocumento(.tex)
\end{verbatim}

Es importante destacar que el fichero \verb|.aux| que se pasa a bibtex es el
resultado de la compilaci�n del documento \verb|.tex|, y no el fichero de la
base de datos en s� (�ste ser� encontrado por la herramienta gracias a que
est�a incluido en el anterior por medio del comando \verb|\bibliography|).
Una vez que se sabe c�mo generar la bibliograf�a, se debe probar a generar
una listado de referencias bibliogr�ficas en el documento actual.


\par Esto es un ejemplo de cita: \cite{armistice}

\nocite*

%%Para indicar el uso de bibliograf�a
\bibliography{bibliografia}
\bibliographystyle{alpha}

\end{document}
