\chapter{Tablas}
\noindent
En algunos documentos puede ser necesario incluir una tabla que ilustre o aclare
nuestra narraci�n.\par
El entorno que permite describir tablas es:\newline

\verb|{\begin{tabular}[caracter�sticas de la tabla]...\end{tabular}|
%\begin{verbatim}
%{\begin{tabular}[caracter�sticas de la tabla]...\end{tabular}
%\end{verbatim}
, donde\newline

\verb|[caracter�sticas de la tabla]|podr� contener lo siguiente:

\begin{itemize}
	\renewcommand{\labelitemi}{$\bullet$}
	\item ``\verb|l|'', para describir una columna con texto justicado a la
	izquierda
	\item ``\verb|r|'', para describir una columna con texto justicado a la derecha
	\item ``\verb|c|'', para describir una columna con texto centrado
	\item ``\verb|p{ancho}|'', para una columna que contenga texto con saltos de
	l�nea
	\item ``\verb|\|'', para una columna que queremos que vaya entre una l�nea
	vertical
\end{itemize}

Cada filla de una tabla se separa de la siguiente mediante una indicaci�n de
salto de l�nea ``\verb|\\|".\par
Los datos de una columna se separan de los de la siguiente utilizando el
car�cter ``\&''.\par
Veamos un ejemplo:\newline

\begin{tabular}{lcc}
Descriptive levels & Precision & Recall \\
General & 99.7\% & 99.4\% \\
Document specific & 96.8\% & 96.3\% \\
Proper noun & 94.4\% & 99.1\% \\
\end{tabular}
\newline

El mismo ejemplo con separadores de l�nea vertical entre las columnas:\newline

\begin{tabular}{|l|c|c|}
Descriptive levels & Precision & Recall \\
General & 99.7\% & 99.4\% \\
Document specific & 96.8\% & 96.3\% \\
Proper noun & 94.4\% & 99.1\% \\
\end{tabular}
\newline

Si queremos separar unas filas de otras mediante l�neas horizontales, debemos
utilizar \verb|\hline|. \newline

\begin{tabular}{|l|c|c|}
\hline
Descriptive levels & Precision & Recall \\
General & 99.7\% & 99.4\% \\
Document specific & 96.8\% & 96.3\% \\
Proper noun & 94.4\% & 99.1\% \\
\hline
\end{tabular}
\newline

M�s l�neas
\newline

\begin{tabular}{|l|c|c|}
\hline
Descriptive levels & Precision & Recall \\
\hline
General & 99.7\% & 99.4\% \\
\hline
Document specific & 96.8\% & 96.3\% \\
\hline
Proper noun & 94.4\% & 99.1\% \\
\hline
\end{tabular}
\newline

Algunas l�neas m�s
\newline

\begin{tabular}{||l|c|c||}
\hline \hline
Descriptive levels & Precision & Recall \\
\hline \hline
General & 99.7\% & 99.4\% \\
\hline
Document specific & 96.8\% & 96.3\% \\
\hline
Proper noun & 94.4\% & 99.1\% \\
\hline \hline
\end{tabular}
\newline

Queremos una tabla centrada
\newline
\begin{center}
\begin{tabular}{||l|c|c||}
\hline \hline
Descriptive levels & Precision & Recall \\
\hline \hline
General & 99.7\% & 99.4\% \\
\hline
Document specific & 96.8\% & 96.3\% \\
\hline
Proper noun & 94.4\% & 99.1\% \\
\hline \hline
\end{tabular}
\end{center}



Para agrupar varias columnas en una se utiliza \verb|\multicolumn|. En el
siguiente ejemplo se agrupan dos columnas en una:
\newline

\begin{tabular}{||l||c|c||c|c||} \hline\hline
& \multicolumn{2}{c||}{Spanish} &
\multicolumn{2}{c||}{English}\\ \hline
Descriptive levels & Precision & Recall & Precision & Recall \\
\hline\hline General & 99.7\% & 99.4\% & 98.6\% & 98.5\% \\
\hline Document specific & 96.8\% & 96.3\% & 95.8\% & 95.6\% \\
\hline Proper noun & 94.4\% & 99.1\% & 98.8\% & 99.8\% \\ \hline
\hline
\end{tabular}
\newline

Las tablas que se han definido hasta ahora aparec�aan en la posici�n en la
que se han escrito en el documento \verb|.tex|.

\section{Referencias a tablas}
\noindent
Sin embargo, en ocasiones puede no interesar incluir una tabla en el discurso de
una narraci�n, sino hacer una referencia a la tabla que puede estar ubicada en
otra parte del documento. 
En este caso la tabla (entorno tabular)deber�a estar incluida dentro de un
entorno \verb|\begin{table}[ubicacion]...\end{table}| como en el ejemplo
siguiente:

\begin{verbatim}
\begin{table}[h]
\small \centering
\begin{tabular}{||l||c|c||c|c||} \hline\hline
& \multicolumn{2}{c||}{Spanish} &
\multicolumn{2}{c||}{English}\\ \hline
Descriptive levels & Precision & Recall & Precision & Recall \\
\hline\hline General & 99.7\% & 99.4\% & 98.6\% & 98.5\% \\
\hline Document specific & 96.8\% & 96.3\% & 95.8\% & 95.6\% \\
\hline Proper noun & 94.4\% & 99.1\% & 98.8\% & 99.8\% \\ \hline
\hline
\end{tabular}
\end{table}
\end{verbatim}

\begin{table}[h]
\small \centering
\begin{tabular}{||l||c|c||c|c||} \hline\hline
& \multicolumn{2}{c||}{Spanish} &
\multicolumn{2}{c||}{English}\\ \hline
Descriptive levels & Precision & Recall & Precision & Recall \\
\hline\hline General & 99.7\% & 99.4\% & 98.6\% & 98.5\% \\
\hline Document specific & 96.8\% & 96.3\% & 95.8\% & 95.6\% \\
\hline Proper noun & 94.4\% & 99.1\% & 98.8\% & 99.8\% \\ \hline
\hline
\end{tabular}
\end{table}

\par
Se ha intentado que LATEX la muestre en la posici�n en la que la se ha
escrito mediante \verb|\begin{table}[h]|. Si no hubiese cabido en esta parte de
la p�gina, \LaTeX la habr�a ubicado en la siguiente p�gina. Los valores del
argumento de ubicaci�n pueden ser:

\begin{itemize}
	\renewcommand{\labelitemi}{$\bullet$}
	\item \verb|"h":| aqu� o lo m�s pr�ximo posible donde se encuentra declarada
	\item \verb|"t":|'' en la parte superior de una p�gina
	\item \verb|"p":|'' en la parte inferior de una p�gina
	\item \verb|"b":|'' en una p�gina que s�lo contenga tablas y figuras

\end{itemize}

En estos casos conviene que en la narraci�n se haga referencia a la tabla.
Esto se hace con \verb|~\ref{tabla1}|, escribiendo dentro del entorno de la
tabla \verb|\label{tabla1}|.
%%Esto es una referencia a la tabla: ~\ref{tabla1}

\begin{verbatim}
\begin{table}
\begin{tabular}{||l||c|c||c|c||} \hline\hline
& \multicolumn{2}{c||}{Spanish} &
\multicolumn{2}{c||}{English}\\ \hline
Descriptive levels & Precision & Recall & Precision & Recall \\
\hline\hline General & 99.7\% & 99.4\% & 98.6\% & 98.5\% \\
\hline Document specific & 96.8\% & 96.3\% & 95.8\% & 95.6\% \\
\hline Proper noun & 94.4\% & 99.1\% & 98.8\% & 99.8\% \\ \hline
\hline
\end{tabular}
\label{tabla1}
\caption{Results of description levels encoding}
\end{table}
\end{verbatim}


\par
Suele ser conveniente que la tabla tenga un t�tulo (pie de tabla). Esto
permite que se aclare su contenido independientemente de donde aparezca. El
t�tulo se a�nade con \verb|\caption{...}|.
\par

Se puede ver un ejemplo en la tabla ~\ref{tabla1}
%%Duda: Problema con posicionamiento de esta tabla. He probado con otro
% valores, pero no hay manera.
\begin{table}[h]
\begin{tabular}{||l||c|c||c|c||} \hline\hline
& \multicolumn{2}{c||}{Spanish} &
\multicolumn{2}{c||}{English}\\ \hline
Descriptive levels & Precision & Recall & Precision & Recall \\
\hline\hline General & 99.7\% & 99.4\% & 98.6\% & 98.5\% \\
\hline Document specific & 96.8\% & 96.3\% & 95.8\% & 95.6\% \\
\hline Proper noun & 94.4\% & 99.1\% & 98.8\% & 99.8\% \\ \hline
\hline
\end{tabular}
\label{tabla1}
%\begin{center}
\caption{Results of description levels encoding}
%\end{center}
\end{table}

\par
Las referencias no s�lo se utilizan con tablas, sino que se puede hacer
referencia a una figura, secci�n o cap�tulo de un documento.
