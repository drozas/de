\chapter{�ndices}

\section{�ndice de contenidos}
\noindent
La creaci�n de �ndices en \LaTeX es totalmente autom�tica, lo cual supone una
gran ventaja ya que el usuario no tiene la necesidad de ir revisando los n�meros
de p�gina de cada capitulo.
Para generar un �ndice de contenidos se utiliza el comando
\verb|\tableofcontents| en el punto del documento donde se quiere que aparezca
el �ndice. Ser�a necesario compilar el fichero \verb|.tex| dos veces para que se
enlace bien y genere de forma correcta el �ndice.

\section{�ndice de figuras y tablas}
Para generar un �ndice de figuras se utiliza el comando \verb|\listoffigures| y
para generar un �ndice de tablas el comando \verb|\listoftables|. Se da la
orden en el punto donde se quiere que aparezca el �ndice correspondiente.
Un ejemplo t�pico para generar �ndices en las p�ginas iniciales es el
siguiente:

\begin{verbatim}
\begin{document}
\maketitle
\newpage % Salto de p�gina
\tableofcontents % indice de contenidos
\newpage % Salto de pagina
\listoffigures % Indice de figuras
\newpage % Salto de pagina
\listoftables % Indice de tablas
\newpage % Salto de pagina
...
\end{document}
\end{verbatim}
Como en el cap�tulo 2 se ha creado al menos una tabla con t�tulo se pide
generar un �ndice de tablas que tendr�a una �nica entrada.
%%No se agrega autom�ticamente, pero consta en el enunciado
%\newpage