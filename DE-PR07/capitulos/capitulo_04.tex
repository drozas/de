\chapter{Variaciones en el encabezado y en el pie de p�gina}
\noindent
Por defecto hay un estilo para el encabezado y pie de p�gina. Por ejemplo,
en esta pr�ctica se puede ver que por defecto aparece un encabezado para las
p�ginas en may�sculas. Pero, �si lo queremos en min�sculas por ejemplo, qu�
se deber�a hacer?
Existe un paquete llamado \verb|fancyhdr| que permite que podamos configurar
tanto el encabezado como el pie de p�gina a nuestro gusto. Se utilizar�a el
paquete como sigue:

\begin{verbatim}

\usepackage{fancyhdr}
\pagestyle{fancy} ...
\fancyhf{} % borrar todos los ajustes
% En lo siguiente, fancyhead sirve para configurar la cabecera, fancyfoot para
% configurar el pie de p�gina.
% Justificaci�n: C=centered, R=right, L=left, (nada)=LRC
% P�gina: O=odd, E=even, (nada)=OE
\fancyhead[RO,LE]{Cabecera1} \fancyhead[LO,RE]{Cabecera2}
\fancyfoot[LO,CE]{Pie1} \fancyfoot[RO,CE]{Pie2} ...

\end{verbatim}

Si en alguna p�gina espec�fica se prefiere aplicar un estilo concreto se puede
usar \verb|\thispagestyle{arg}|, con \verb|arg={fancy}| o \verb|arg={plain}| o
\verb|arg={empty}|, dependiendo de si se quiere aplicar el estilo especial, el
estilo por defecto o ninguno (sin cabecera ni pie), respectivamente.
Con ``\verb|\Cabecera1|'', ``\verb|\Cabecera2|'', ``\verb|\Pie1|'', ``\verb|\Pie2|'', ... se
puede especificar el n�mero/nombre del cap�tulo/secci�n, etc. Para ello, hay
que tener en cuenta que:

\begin{itemize}
	\renewcommand{\labelitemi}{$\bullet$}
	\item \verb|\leftmark| informaci�n de nivel superior (por ejemplo, cap�tulo en la clase
de documento \verb|book|)
	\item \verb|\righttmark| informaci�n de nivel inferior (p.e.: secci�n en clase
	book)
\end{itemize}

Estos comandos se introducen en \verb|\fancyhead| o \verb|\fancyfoot| seg�n se
requiera, por ejemplo, en un documento clase book,
\verb|\fancyhead[LO,RE]{\leftmark}| indica que debe aparecer el nombre del
cap�tulo en la parte izquierda de la cabecera si es p�gina impar, y en la
derecha si es p�gina par. Para controlar c�mo se representan los cap�tulos,
secciones, etc., en la cabecera/pie del documento, se redefinen los comandos
\verb|\chaptermark|,\verb|\sectionmark|,\verb|\subsectionmark|, etc. tras la
llamada a \verb|\pagestyle{fancy}|, por ejemplo: El n�mero de p�gina es
\verb|\thepage|. Puede aparecer en \verb|\fancyhead| o \verb|\fancyfoot|, seg�n
se quiera; por ejemplo, \verb|\fancyfoot[C]{\thepage}| indica que el n�mero de
p�gina aparecer�a centrado en el pie de todas las p�ginas. En nuestro caso
queremos que en el encabezado, en las p�ginas impares, en la parte izquierda
aparezca ``Pr�ctica 7'' y en las p�ginas pares, en la parte derecha
aparezca ``Documentaci�n Electr�nica''. Adem�s, los n�meros de p�gina
aparecer�an en el pie de p�gina y centrados. El resultado de estos cambios se
puede ver en el archivo ``practica7Variacion.pdf''. Escribid la sintaxis para
conseguir lo anterior y comprobar que el encabezado y pie de p�gina ha cambiado.
